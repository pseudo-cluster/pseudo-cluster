\documentclass[12pt]{article}
\usepackage{a4wide}
\usepackage[utf8]{inputenc}
\usepackage[russian]{babel}
\usepackage[dvips]{graphicx, color}
\usepackage{epstopdf}
\usepackage{amsmath}
\usepackage{amsfonts}
\usepackage{amssymb}
\usepackage{amsthm}
\usepackage{tikz}
\usepackage{pgfplots}

\begin{document}

\thispagestyle{empty}

\section{Описание используемых метрик}
	Имеется набор задач $T$, каждая из которых запускалась на вычислительном кластере, управляемом системой ведения очередей Slurm.
	Пусть $U = \{u | \exists t \in T: t_{user} = u\}$ --- множество всех пользователей, которые ставили задачи из набора задач $T$,
	$T_u$ множество задач из набора задач $T$, которые были поставлены пользователем $u \in U$. 
	
	По этому набору вычисляются числовые характеристики (более подробно описаны далее):
	\begin{itemize}
		\item Среднее время ожидания в очереди (для каждого пользователя)
		\item Средняя загруженность вычислительной системы
		\item Среднее отклонение количества запрашиваемых процессоров (каждым пользователем\\по дням\\в общем)
		\item Разумность запрашиваемого времени исполнения
	\end{itemize}
	
	У каждой задачи $t \in T$ есть несколько параметров, нас будут интересовать следующие:
	\begin{itemize}
		\item $t_{time\_start}$ --- дата и время, когда задача начала выполнение
		\item $t_{time\_end}$ --- дата и время, когда задача закончила выполнение
		\item $t_{time\_queue}$ --- дата и время, когда задача была поставлена в очередь
		\item $t_{required\_cpus}$ --- запрошенное пользователем количество процессоров для выполнения задачи
		\item $t_{time\_limit}$ --- запрошенное пользвателем ограничение по времени исполнения задачи
		\item $t_{status}$ --- статус запуска задачи
		\item $t_{user}$ --- пользователь, который поставил задачу
	\end{itemize}
	
	Далее под вычитанием двух дат подразумевается количество минут, которое прошло от уменьшаемой даты до вычитаемой даты.
	
	\subsection{Среднее время ожидания в очереди}
	Среднее время ожидания в очереди для задач пользователя $u \in U$ считается следующим образом:
	$$ \frac{\sum_{t \in T, t_{user}=u} {(t_{time\_start} - t_{time\_queue})}}{|\{t|t \in T, t_{user} = u\}|} $$
	
	\subsection{Средняя загруженность вычислительной системы}
	Пусть $d$ --- некая единица измерения времени (например, два часа), указанная в минутах. Также обозначим $len(t_1, t_2) = (t_2 - t_1) / d$,
	$time_{min} = \min \{t_{time\_start}| t \in T\}$, $time_{max} = \max \{t_{time\_end}| t \in T\}$.
	
	Средняя загруженность $R$ измеряется в процессорах. Она призвана показать, насколько много процессоров в среднем (на единицу времени $d$) было
	 использовано за весь период времени, когда запускались задачи из набора задач $T$. Формула для вычисления:
	 $$ R = \frac{\sum_{t \in T} {t_{required\_cpus} * len(t_{time\_start}, t_{time\_end})}}{len(time_{min}, time_{max})}$$
	 
	\subsection{Среднее отклонение количества запрашиваемых процессоров}
	Обозначим за $RC_u = \{t_{required\_cpus}|t \in T_u\}$, $median_u$ 
	---	медиана множества $RC_u$. 
	
	Тогда среднее отклонение количества запрашиваемых конкретным пользователем $u$ процессоров вычисляется так:
	$$ \frac{\sum_{cpu \in RC_u} (cpu - median_u)}{|RC_u|} $$
	
	Аналогично вычисляются общее отклонение запрашиваемых процессоров, а также отклонение по дням.
	
	\subsection{Разумность запрашиваемого времени исполнения}
	Для каждого пользователя $u \in U$ считается множество задач, которые успешно выполнились: $SC_u = \{t \in T_u| t_{status} = "completed"\}$.
	Тогда разумность запрашиваемого времени исполнения для данного пользователя вычисляется так:
	$$ \frac{\sum_{t \in SC_u} \frac{t_{time\_end} - t_{time\_start}}{t_{time\_limit}}}{|SC_u|} $$
 
        
\end{document}
